\renewcommand{\abstractnamefont}{\normalfont\Large\bfseries}
%\renewcommand{\abstracttextfont}{\normalfont\Huge}

\begin{abstract}
% \hskip7mm

% \begin{spacing}{1.3}

    Le domaine maritime, avec tous les enjeux qu'il comporte, présente un grand nombre
    de problématiques qui peuvent être résolues par les nouvelles technologies.
    On trouve parmi celles-ci le routage, l'exploration des fonds marins ou encore la surveillance. \newline

    MaxSea International, en association avec Furuno, fait partie des acteurs qui fournissent
    des solutions innovantes aux professionnels de ce domaine.
    L'entreprise, localisée à Bidart dans le Pays Basque, compte 70 employés dont une majorité de développeurs.
    Elle est à l'origine des logiciels de la gamme TimeZero, qui comptent, entre autres :
    \begin{itemize}
        \item{\textbf{TZ Professional}, qui permet le routage des navires de pêche et commerciaux,
        l'analyse de la bathymétrie ou encore la gestion d'appareils d'acquisition ;}
        \item{\textbf{TZ Maps}, une collection de cartes précises aux formats vectoriel et
        raster dans le monde entier ;}
        \item{\textbf{TZ iBoat}, l'application iOS pour créer des itinéraires de plaisance.}
    \end{itemize}

    Le logiciel qui bénéficiera de notre travail est \textbf{TZ Coastal Monitoring},
    destiné aux ports et zones industrielles côtières.
    Il permet la surveillance des navires par différents moyens, et contient
    un module de gestion des caméras de surveillance.

    Les principaux éléments permettant de détecter un bateau sont le radar et l'AIS.
    Le radar fonctionne grâce à des ondes radio courtes
    (source: \href{https://info.furuno.fr/comment-fonctionne-le-radar-pour-bateau}{Furuno Radar})
    et permet de détecter toutes sortes d'objets, dont des bateaux.
    L'AIS, acronyme de Automatic Identification System
    (source: \href{https://www.imo.org/en/OurWork/Safety/Pages/AIS.aspx}{IMO AIS})
    est un système de balise embarquée
    permettant d'éviter les collisions entre les bateaux, et d'indiquer des informations
    aux infrastructures côtières.
    Ces deux systèmes, bien qu'étant performants, ont des défauts indéniables.
    Le radar par exemple, n'apporte pas d'information précise sur la nature de l'objet détecté,
    souffre d'une latence importante et est sensible aux conditions météorologiques.
    L'AIS, quant à lui, peut facilement être désactivé par l'équipage.
    C'est pour cela que MaxSea International s'intéresse à la détection automatique d'objets,
    en particulier par réseaux de neurones.
    Un tel système permettrait de nombreuse fonctionnalités nouvelles, en profitant du matériel déjà présent,
    donc à moindre coût pour les clients ;
    pilotage automatique des caméras, alertes en temps réel, ou encore enregistrement d'images précises
    lors d'incidents.

    Comme il n'existe pas de proposition spécifiquement conçue pour le métier maritime,
    l'entreprise a décidé de construire leur propre solution.
    Pour accomplir cette mission, nous avons mis en place un pipeline de machine learning qui permet d'aller
    de la collecte de datasets
    jusqu'à la production d'un modèle optimisé, prêt à être utilisé.
    Ce modèle, YOLOX, fait partie des modèles open source les plus performants à ce jour. 
    Les solutions apportées par le stage sont une base de données de plus de 120 000 images de bateaux,
    des systèmes de preprocessing, un environnement d'entraînement du modèle YOLOX,
    et différents modèles optimisés pour les processeurs cibles. \newline

    Les chapitres suivants présentent en détails les éléments précédemment mentionnés, le travail réalisé et les conclusions auxquelles nous avons abouti.

% \end{spacing}
\end{abstract}
\thispagestyle{empty}
