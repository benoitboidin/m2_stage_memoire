\chapter{Cahier des charges}

L'objectif principal du stage était de créer un système de détection de navires pour enrichir la fonction de 
vidéo-surveillance de Coastal Monitoring. Cet objectif peut être décomposé en plusieurs éléments : 

\section{Précision de la détection}

La précision est le point principal de ce projet. Il est important que, lorsque le 
système détecte un objet, cet objet soit en effet un navire : du point de vue du client, 
les fausses alertes entravent grandement la qualité perçue, même si elles sont rares. 
Il faut donc que le modèle soit le plus "prudent" possible, tout en étant quand même
capable de détecter les navires.

\section{Vitesse de traitement}

La vitesse de traitement est un autre point important. Le système doit être capable 
de détecter les navires en temps réel, c'est-à-dire que le temps de traitement d'une image
doit être assez rapide pour être utile en vidéo (environ 30 images par seconde).
Il est nécessaire de préciser que ces performances doivent être réalisées sur 
des machines similaires à celles utilisées par les clients, c'est à dire sans 
carte graphique dédiée, et avec des processeurs n'était pas nécessairement de 
dernière génération. 

\section{Visualisation}

Pour permettre l'analyse qualitative des résultats, il est nécessaire de pouvoir visualiser, 
non seulement les résultats de la détection, mais aussi les datasets qui serviront à entraîner. 
Pour ce faire, j'ai utilisé le logiciel FiftyOne tout au long du développement. 

\section{Facilité d'utilisation\label{facilite_utilisation}}

MaxSea ne comptant pas de développeur familier avec ces technologie, 
il est important que le système soit facile à utiliser et à maintenir.
J'ai pour cela utilisé, comme le stagiaire précédent, des Jupyter Notebooks. 
Ce projet n'étant pas encore destiné à être intégré dans le produit final,
les notebooks sont un bon compromis entre ergonomie et rapidité de mise en place. 
Ceux-ci permettent d'exécuter pas à pas des scripts Python, 
et la recherche de bug est plus aisée. 