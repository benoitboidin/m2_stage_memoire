\chapter{Cahier des charges}

L'objectif principal du stage était d'entraîner un modèle de détection
de navires pour enrichir la fonction de vidéo-surveillance de Coastal Monitoring.
Cela améliorerait le suivi de cibles, en complément du radar qui ne permet 
pas de connaître la nature de l'écho détecté.
Cet objectif peut être décomposé en plusieurs éléments, détaillés ci-dessous.

\section{Précision de la détection}

La précision est le point clef de ce projet. Il est important que, lorsque le
système détecte un objet, cet objet soit en effet un navire : du point de vue du client,
les fausses alertes entravent grandement la qualité perçue, même si elles sont rares.
Il faut donc que le modèle soit le plus "prudent" possible, tout en étant quand même
capable de détecter les navires.

\section{Vitesse de traitement}

La vitesse de traitement est un autre point important. Le système doit être capable
de détecter les navires en temps réel, c'est-à-dire que le temps de traitement d'une image
doit être assez rapide pour être utile en vidéo (environ 30 images par seconde).
Il est nécessaire de préciser que ces performances doivent être réalisées sur
des machines similaires à celles utilisées par les clients, c'est à dire sans
carte graphique dédiée, et avec des processeurs n'étant pas nécessairement de
dernière génération.

\section{Visualisation et supervision}

Afin de partager nos résultats à l'équipe, d'explorer les datasets et d'observer les résultats
d'inférence, nous avons choisi d'utiliser FiftyOne, un outil open source.\\ 

Pour le suivi des performances pendant l'entraînement, nous avons utilisé Weight\&Biaises. 
Cet outil présente l'avantage d'être en ligne, ce qui permet de suivre les entraînement 
à distance et facilement partager les résultats. Après discussion, nous avons décidé de 
changer au profit de TensorBoard, qui permet une meilleure protection des données.

\section{Facilité d'utilisation\label{facilite_utilisation}}

Il est important que le système soit facile à utiliser et à maintenir.
Nous avons pour cela utilisé, comme le stagiaire précédent, des Jupyter Notebooks.
Ce projet n'étant pas encore destiné à être intégré dans le produit final,
les notebooks sont un bon compromis entre ergonomie et rapidité de mise en place.
Ceux-ci permettent d'exécuter pas à pas des scripts Python,
et la recherche de bug est plus aisée.
