\chapter{Conclusions}

\section{Solutions retenues}

Parmi tous les essais que nous avons faits, augmenter le nombre d'images dans le dataset 
d'entraînement semble être le moyen le plus efficace d'améliorer la précision et le 
rappel du modèle. 

En revanche, il est important que ces images soient le plus hétérogène possible, 
pour éviter le surapprentissage. Le seul filtre qui nous a permis d'obtenir de 
meilleurs scores en diminuant la taille du dataset est le filtre de similarité. 

Les autres filtres, basés sur les tailles d'objets ou la qualité d'image, 
n'ont produit que des résultats négatifs. 

Des recherches sont encore en cours pour trouver le modèle qui apporte 
le meilleur compromis entre précision et vitesse d'exécution : lors 
de l'intégration dans le logiciel TimeZero Coastal Monitoring, il sera couplé 
à un logiciel de suivi de cibles qui dépend du temps d'inférence. 

\section{Bénéfices du stage}

Les bénéfices du stage se mesurent principalement grâce aux scores obtenus 
par YOLOX : la précision et le rappel ont respectivement été augmentés de 85\%
et 40\%. \\

La documentation produite a également été d'une grande aide, car elle a apporté à l'équipe une compréhension 
beaucoup plus fine de tout l'environnement d'apprentissage automatique. \\

Enfin, ce projet pourra être repris au-delà du stage, pour permettre des 
entraînements avec de nouvelles images. \\

De notre côté, ce stage a eu un impact très positif sur nos compétences. 
En étant responsable de toute la chaîne de création d'un modèle d'apprentissage 
automatique, au sein d'une entreprise produisant des algorithmes plus "classiques", 
nous avons eu la chance de maîtriser toutes les étapes, et de grandement affiner
nos capacités de recherches.

\section{Développement durable}

Mes travaux sur l'algorithme de détection de navires ont nécessité l'utilisation 
de cartes graphiques, qui consomment beaucoup d'énergie pour fonctionner. 
Dans le cadre de ce projet, j'ai effectué environ 60 entraînements du modèle, 
ce qui a consommé une quantité importante d'énergie électrique. 
Selon les calculs effectués (\textit{source : }\url{http://calculator.green-algorithms.org}), 
cela représente un impact environnemental notable, 
avec un total de 762,83 kg CO2e émis au cours de ces entraînements. 
Cette constatation soulève des questions sur l'impact global de mes travaux 
et le besoin de trouver des solutions pour réduire l'empreinte écologique de ce type d'applications. 
Cependant, il est également important de noter que cette technologie a le potentiel 
d'améliorer la sécurité et la surveillance dans les ports et les zones côtières, 
réduisant ainsi le risque d'accidents et de dommages causés par les navires. 
Il s'agit donc d'un exemple de compromis entre des enjeux concurrents qui 
nécessite une attention particulière pour minimiser l'impact négatif et maximiser les avantages positifs.