\chapter{Conclusions}

\section{Solutions retenues}

\section{Développement durable}

Mes travaux sur l'algorithme de détection    de navires nécessitent l'utilisation 
de cartes graphiques, qui consomme beaucoup d'énergie pour fonctionner. 
Dans le cadre de ce projet, j'ai effectué environ 60 entraînements du modèle, 
ce qui a consommé une quantité importante d'énergie électrique. 
Selon les calculs effectués (\textit{source : }\url{http://calculator.green-algorithms.org}), 
cela représente un impact environnemental notable, 
avec un total de 762,83 kg CO2e émis au cours de ces entraînements. 
Cette constatation soulève des questions sur l'impact global de mes travaux 
et le besoin de trouver des solutions pour réduire l'empreinte écologique de ce type d'applications. 
Cependant, il est également important de noter que cette technologie a le potentiel 
d'améliorer la sécurité et la surveillance dans les ports et les zones côtières, 
réduisant ainsi le risque d'accidents et de dommages causés par les navires. 
Il s'agit donc d'un exemple de compromis entre des enjeux concurrents qui 
nécessite une attention particulière pour minimiser l'impact négatif et maximiser les avantages positifs.