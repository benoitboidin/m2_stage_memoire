\chapter{Introduction et contexte}


\section{Présentation de l'entreprise}

Il y a 35 ans, Brice Pryszo a fondé MaxSea International et a créé le premier logiciel de navigation embarqué
permettant de stocker des cartes marines sur un ordinateur. Depuis, l'entreprise n'a cessé de proposer
des solutions toujours plus innovantes pour les professionnels de la mer, dans plus de 25 pays.
Ses clients aussi bien les plaisanciers que les pêcheurs ou la marine marchande. \\
Elle propose notamment les solutions de navigation TimeZero Navigator (destiné aux plaisanciers)
et TimeZero Professional (destiné aux professionnels de la mer), appuyées par TimeZero Maps,
une cartographie marine de haute qualité en raster et, depuis peu, en vecteur.
Ces cartes sont également accessibles sur l'application iOS TimeZero iBoat.
Tous ces produits profitent de leur service météo, qui donne accés à des précisions
météorologiques très complètes, fournies par les modèles météo les plus fiables.\\

Ces technologies s'appuient sur des appareils d'acquisition haut de gamme, plus particulièrement
ceux fabriqués par Furuno, partenaire principal de la société depuis 2007. Cette collaboration
bénéficie à TimeZero Coastal Monitoring, qui permet de surveiller les zones côtières. \\

C'est sur ce dernier produit que j'ai effectué mon stage de fin d'études, au sein de l'équipe
de recherche et développement. Plus particulièrement, j'ai développé une fonctionnalité
de détection de navires, qui permettra de compléter les informations acquise par le radar
et l'AIS, dont on peut voir un exemple sur la figure ci-après \ref{fig:radar}.

\begin{figure}[H]
    \centering
    \includegraphics[width=0.8\textwidth]{./img/ports-harbors-cameras1.jpg}
    \caption{Exemple de vue radar, AIS et camera dans TimeZero Coastal Monitoring}
    \label{fig:radar}
\end{figure}

L'équipe que j'ai rejoint était composée de 2 personnes, dont Ronan Golhen, mon tuteur de stage et
directeur technique, et Victor Opter, développeur logiciel. L'entreprise ne comptant parmi
ses effectifs aucun développeur en apprentissage automatique, j'ai travaillé en autonomie sur le projet,
tout en profitant de la grande expertise métier de mon tuteur, qui connaît non seulement
la partie logiciel, mais qui a de surcroit une grande expérience en navigation. \\

\section{Début du projet}

Ce projet a commencé en 2023, avec un premier stagiaire qui a travaillé sur la détection de navires. \\

Ses travaux comptent tout d'abord le choix d'un modèle de machine learning nommé YOLOX
(\textit{voir }\ref{yolox}).
Ce choix a été guidé par la recherche de performances, aussi bien du côté de la précision
que de celui de la rapidité, et par des contraintes légales. Ce modèle étant sous
licence Apache 2.0, il est possible de l'utiliser de façon commerciale.\\

Après avoir choisi ce modèle, il a été décidé qu'une phase d'entraînement était nécessaire.
Il a donc commencé à récolter des datasets, réunissant environ 8000 images de navires.
Trois entraînements on été réalisés en utilisant des machines distantes via Google Collab. \\

Les résultats du projet comportaient néanmoins quelques limites. Entre autres, les scripts
contenait un grand nombre de chemins d'accès non relatifs, ce qui rendait
l'exécution. De plus, la documentation très succinte ne permettait pas
de connaître les paramètres exacts utilisés pour l'entraînement. De plus, l'achat de matériel
spécialisé a rendu obsolète le code relatif à l'utilisation du cloud.\\

Après avoir pris connaissance de ces travaux, j'ai donc entrepris d'utiliser les outils choisis
l'année dernière, en écrivant des scripts plus modulaires et en documentant plus précisément
les étapes de l'entraînement.
